\documentclass[11pt]{article}

\usepackage{etex}
\usepackage{hyperref}
\usepackage{texyousei}

\lhead{15-467/667}
\rhead{Assignment 3}

\begin{document}

\thispagestyle{empty}
\maketitle{Pausch Bridge MIDI Visualizer}{Spring 2016, Group 3}

\setlength{\leftmargini}{0.3in}

\section{Abstract}

This document describes a MIDI-based music visualizer that can be run
on the Pausch Bridge. The lights on the bridge mimic the keys of a
piano, with different pitches being mapped to corresponding positions
along the length of the bridge.

\section{Running the Project}

\subsection{Dependencies}

The project is in Python, and requires two additional Python
libraries: \texttt{pysmf} and \texttt{pyOSC}. Additionally, the
visualizer requires \texttt{PyOpenGL}. All of them can be installed
through pip. (You may need to \texttt{pip install pyOSC --pre} if the
problem with the version numbers hasn't been fixed yet.)

\subsection{Overview and TLDR}

The main project is contained in the \texttt{midi/} folder. Suppose we
have a file \texttt{foo.mid}. If we want to turn it into a light show
and view it using the local visualizer, we need to complete the
following steps:

\begin{enumerate}

\item Convert \texttt{foo.mid} into \texttt{foo.mid.notelist}:
  \texttt{python SongReader.py foo.mid}

\item Start the server: \texttt{python BridgeServer\_GUI.py}

\item Send the notes to the server: \texttt{python SendMidi.py
  foo.mid.notelist 0 localhost}

\item (Optional) Play an audio file that will be synced to the light
  show.

\end{enumerate}

If we want to play \texttt{foo.mid} on the bridge instead of viewing
it locally:

\begin{enumerate}

\item Convert the file as before.

\item SSH to the bridge and start the server there: \texttt{python
  BridgeServer.py}

\item Either from your local computer, or from the bridge itself, send
  the notes: \texttt{python SendMidi.py foo.mid.notelist 0
    pbridge.adm.cs.cmu.edu}

\item (Optional) Get some very loud speakers and sync your audio to
  the show.

\end{enumerate}

\subsection{Converting MIDI}

The MIDI standard is generally too messy to convert directly into
lights on the bridge, so we convert it to an intermediary format
\texttt{.notelist}, which strips out all of the cruft of typical MIDI
files and leaves only basic information about notes, such as their
times, pitches, and durations. The script \texttt{SongReader.py} will
perform this conversion. It can be used with the command

\begin{center}
\texttt{python SongReader.py foo.mid}
\end{center}

where \texttt{foo.mid} is the MIDI file we want to convert. It will
then create \texttt{foo.mid.notelist} in the same directory, which the
rest of the system will be able to use.

\subsection{Starting the Server}

We provide two servers: one (\texttt{BridgeServer\_GUI.py}) that
displays a visualizer on the local machine, which is useful for
testing out different songs, and one (\texttt{BridgeServer.py}) that
is meant to run on the bridge and interface with the lights. Neither
of them take any command-line arguments, so they can be started by
typing

\begin{center}
\texttt{python BridgeServer[\_GUI].py}
\end{center}

The GUI version will initialize itself on localhost by default, while
the actual bridge server will initialize itself on
\texttt{pbridge.adm.cs.cmu.edu}. The servers will only accept packets
sent specifically to the address they are initialized on. Practically,
this means that if you want to stream a light show from the bridge to
itself, you can't send to localhost -- you have to send packets
explicitly to \texttt{pbridge.adm.cs.cmu.edu}.

\subsection{Streaming the Notes}

Once the server is running either locally or on the bridge, you can
begin sending the notes to the bridge. The script to do this is
\texttt{SendMidi.py}, which can be used with the command:

\begin{center}
\texttt{python SendMidi.py <notelist> <wait time> <address>}
\end{center}

Here, \texttt{notelist} is the \texttt{.notelist} file we created in
step 1, \texttt{wait time} is the time in seconds that we should wait
before sending the first note, and \texttt{address} is the network
address where the server is running. In particular, this address
should be localhost if we are running the GUI server, and
\texttt{pbridge.adm.cs.cmu.edu} if we are running the server on the
bridge.

The wait time is useful in case we want to sync the light show with an
audio file (e.g. an mp3), but the audio file has some leading
silence. By specifying a short delay, we could delay the start of the
light show until the start of the actual audio, assuming that we start
playing the song at the exact same time that we start running
\texttt{SendMidi.py}. (However, the better solution is just to trim
the audio file so that there isn't any leading silence.)

So, if we want to use \texttt{foo.mid.notelist} as the light show and
we already have the server running on the bridge, we would type
\texttt{python SendMidi.py foo.mid.notelist 0
  pbridge.adm.cs.cmu.edu}. If we simply wanted to preview the show on
our own computer using the visualizer, we would type \texttt{python
  SendMidi.py foo.mid.notelist 0 localhost}.

Note that the script gives a 5 second countdown before starting the
show. You can use this in case you need to start music manually at the
right time.

\subsection{Combined script}

We have a small script \texttt{run.sh} that can sync the start of the
show with the start of an mp3 file. The syntax for this script is

\begin{center}
\texttt{./run.sh <notelist file> <mp3 file> <address> <midi delay>
  <mp3 delay>}
\end{center}

The first three arguments are self-explanatory. The MIDI delay is the
time in seconds to delay before starting to send the MIDI, in case the
MIDI begins immediately but the mp3 has leading silence. The mp3 delay
is the reverse -- the time to delay before starting to play the mp3,
in case the mp3 begins immediately but the MIDI has some leading
silence. (Again, it's better just to edit both so that they start
immediately. If you do, then both arguments can just be 0.)

\section{Conclusion}

If you have questions, email Chris at \texttt{christoy@cs.cmu.edu}.

\end{document}
